% Options for packages loaded elsewhere
\PassOptionsToPackage{unicode}{hyperref}
\PassOptionsToPackage{hyphens}{url}
\PassOptionsToPackage{dvipsnames,svgnames,x11names}{xcolor}
%
\documentclass[
  letterpaper,
  DIV=11,
  numbers=noendperiod]{scrartcl}

\usepackage{amsmath,amssymb}
\usepackage{iftex}
\ifPDFTeX
  \usepackage[T1]{fontenc}
  \usepackage[utf8]{inputenc}
  \usepackage{textcomp} % provide euro and other symbols
\else % if luatex or xetex
  \usepackage{unicode-math}
  \defaultfontfeatures{Scale=MatchLowercase}
  \defaultfontfeatures[\rmfamily]{Ligatures=TeX,Scale=1}
\fi
\usepackage{lmodern}
\ifPDFTeX\else  
    % xetex/luatex font selection
\fi
% Use upquote if available, for straight quotes in verbatim environments
\IfFileExists{upquote.sty}{\usepackage{upquote}}{}
\IfFileExists{microtype.sty}{% use microtype if available
  \usepackage[]{microtype}
  \UseMicrotypeSet[protrusion]{basicmath} % disable protrusion for tt fonts
}{}
\makeatletter
\@ifundefined{KOMAClassName}{% if non-KOMA class
  \IfFileExists{parskip.sty}{%
    \usepackage{parskip}
  }{% else
    \setlength{\parindent}{0pt}
    \setlength{\parskip}{6pt plus 2pt minus 1pt}}
}{% if KOMA class
  \KOMAoptions{parskip=half}}
\makeatother
\usepackage{xcolor}
\setlength{\emergencystretch}{3em} % prevent overfull lines
\setcounter{secnumdepth}{5}
% Make \paragraph and \subparagraph free-standing
\ifx\paragraph\undefined\else
  \let\oldparagraph\paragraph
  \renewcommand{\paragraph}[1]{\oldparagraph{#1}\mbox{}}
\fi
\ifx\subparagraph\undefined\else
  \let\oldsubparagraph\subparagraph
  \renewcommand{\subparagraph}[1]{\oldsubparagraph{#1}\mbox{}}
\fi


\providecommand{\tightlist}{%
  \setlength{\itemsep}{0pt}\setlength{\parskip}{0pt}}\usepackage{longtable,booktabs,array}
\usepackage{calc} % for calculating minipage widths
% Correct order of tables after \paragraph or \subparagraph
\usepackage{etoolbox}
\makeatletter
\patchcmd\longtable{\par}{\if@noskipsec\mbox{}\fi\par}{}{}
\makeatother
% Allow footnotes in longtable head/foot
\IfFileExists{footnotehyper.sty}{\usepackage{footnotehyper}}{\usepackage{footnote}}
\makesavenoteenv{longtable}
\usepackage{graphicx}
\makeatletter
\def\maxwidth{\ifdim\Gin@nat@width>\linewidth\linewidth\else\Gin@nat@width\fi}
\def\maxheight{\ifdim\Gin@nat@height>\textheight\textheight\else\Gin@nat@height\fi}
\makeatother
% Scale images if necessary, so that they will not overflow the page
% margins by default, and it is still possible to overwrite the defaults
% using explicit options in \includegraphics[width, height, ...]{}
\setkeys{Gin}{width=\maxwidth,height=\maxheight,keepaspectratio}
% Set default figure placement to htbp
\makeatletter
\def\fps@figure{htbp}
\makeatother

<link href="fixed-effects_files/libs/panelset-0.2.6/panelset.css" rel="stylesheet" />
<script src="fixed-effects_files/libs/panelset-0.2.6/panelset.js"></script>
\usepackage[sfdefault]{roboto}
\KOMAoption{captions}{tableheading}
\makeatletter
\makeatother
\makeatletter
\makeatother
\makeatletter
\@ifpackageloaded{caption}{}{\usepackage{caption}}
\AtBeginDocument{%
\ifdefined\contentsname
  \renewcommand*\contentsname{Table of contents}
\else
  \newcommand\contentsname{Table of contents}
\fi
\ifdefined\listfigurename
  \renewcommand*\listfigurename{List of Figures}
\else
  \newcommand\listfigurename{List of Figures}
\fi
\ifdefined\listtablename
  \renewcommand*\listtablename{List of Tables}
\else
  \newcommand\listtablename{List of Tables}
\fi
\ifdefined\figurename
  \renewcommand*\figurename{Figure}
\else
  \newcommand\figurename{Figure}
\fi
\ifdefined\tablename
  \renewcommand*\tablename{Table}
\else
  \newcommand\tablename{Table}
\fi
}
\@ifpackageloaded{float}{}{\usepackage{float}}
\floatstyle{ruled}
\@ifundefined{c@chapter}{\newfloat{codelisting}{h}{lop}}{\newfloat{codelisting}{h}{lop}[chapter]}
\floatname{codelisting}{Listing}
\newcommand*\listoflistings{\listof{codelisting}{List of Listings}}
\makeatother
\makeatletter
\@ifpackageloaded{caption}{}{\usepackage{caption}}
\@ifpackageloaded{subcaption}{}{\usepackage{subcaption}}
\makeatother
\makeatletter
\@ifpackageloaded{tcolorbox}{}{\usepackage[skins,breakable]{tcolorbox}}
\makeatother
\makeatletter
\@ifundefined{shadecolor}{\definecolor{shadecolor}{rgb}{.97, .97, .97}}
\makeatother
\makeatletter
\makeatother
\makeatletter
\makeatother
\ifLuaTeX
  \usepackage{selnolig}  % disable illegal ligatures
\fi
\IfFileExists{bookmark.sty}{\usepackage{bookmark}}{\usepackage{hyperref}}
\IfFileExists{xurl.sty}{\usepackage{xurl}}{} % add URL line breaks if available
\urlstyle{same} % disable monospaced font for URLs
\hypersetup{
  pdftitle={Fixed Effects Regression},
  pdfauthor={Andrew Grogan-Kaylor},
  colorlinks=true,
  linkcolor={blue},
  filecolor={Maroon},
  citecolor={Blue},
  urlcolor={Blue},
  pdfcreator={LaTeX via pandoc}}

\title{Fixed Effects Regression}
\author{Andrew Grogan-Kaylor}
\date{2023-11-16}

\begin{document}
\maketitle
\ifdefined\Shaded\renewenvironment{Shaded}{\begin{tcolorbox}[enhanced, sharp corners, frame hidden, boxrule=0pt, interior hidden, borderline west={3pt}{0pt}{shadecolor}, breakable]}{\end{tcolorbox}}\fi

\renewcommand*\contentsname{Table of contents}
{
\hypersetup{linkcolor=}
\setcounter{tocdepth}{3}
\tableofcontents
}
\hypertarget{acknowledgement}{%
\section{Acknowledgement}\label{acknowledgement}}

This presentation of these ideas draws heavily upon the Stata
documentation, although I have changed the notation slightly, and drawn
out a few steps.

\hypertarget{derivation}{%
\section{Derivation}\label{derivation}}

\hypertarget{a-regression-model-with-person-specific-effects}{%
\subsection{A Regression Model With Person Specific
Effects}\label{a-regression-model-with-person-specific-effects}}

We start with our regression equation.

\begin{equation}\protect\hypertarget{eq-regression}{}{y_{it} = \beta_0 + \beta_1 x_{it} + u_{0i} + e_{it}}\label{eq-regression}\end{equation}

\hypertarget{the-basics}{%
\subsection{The Basics}\label{the-basics}}

\hypertarget{intermediate-steps}{%
\subsection{Intermediate Steps}\label{intermediate-steps}}

\hypertarget{intermediate-step-1}{%
\subsubsection{Intermediate Step 1}\label{intermediate-step-1}}

We can sum both sides over the \(t\) time points.

\[\sum_{t} y_{it} = \sum_{t} \beta_0 + \sum_{t} \beta_1 x_{it} + \sum_{t} u_{0i} + \sum_{t} e_{it}\]

\hypertarget{intermediate-step-2}{%
\subsubsection{Intermediate Step 2}\label{intermediate-step-2}}

We can then divide by the number of time points (\(T_i\)).

\[\sum_{t} y_{it} / T_i = \sum_{t} \beta_0 / T_i + \sum_{t} \beta_1 x_{it} / T_i + \sum_{t} u_{0i} / T_i + \sum_{t} e_{it} / T_i\]

\hypertarget{the-between-estimator}{%
\subsection{The Between Estimator}\label{the-between-estimator}}

If Equation~\ref{eq-regression} is true, then the below must also be
true:

\begin{equation}\protect\hypertarget{eq-between}{}{\bar y_i = \beta_0 + \beta_1 \bar x_i + u_{0i} + \bar e_i}\label{eq-between}\end{equation}

This is sometimes called the between estimator.

\hypertarget{the-fixed-effects-estimator}{%
\subsection{The Fixed Effects
Estimator}\label{the-fixed-effects-estimator}}

We can subtract Equation~\ref{eq-between} from
Equation~\ref{eq-regression}.

\begin{equation}\protect\hypertarget{eq-within}{}{y_{it} - \bar y_i = \beta_1 (x_{it} - \bar x_i) + (e_{it} - \bar e_i)}\label{eq-within}\end{equation}

Equation~\ref{eq-within} is the fixed effects (or \emph{within})
estimator.



\end{document}
